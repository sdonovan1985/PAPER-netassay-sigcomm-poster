
\begin{comment}
%\subsection{Behavioral Queries}
\section{Example Behaviors}
\label{sec:examples}
Before going into the design, a few examples follow. This is in Pyretic syntax~\cite{pyretic}, where \tti{>}\tti{>} is sequential composition (operations run in order) and \tti{+} is parallel composition (operations run in parallel). 

\begin{Verbatim}[fontsize=\small]
  traffic_in_bytes(to = 'BobsPC', 
                   content = 'video',
                   timespan = 'monthly') > 10**9 >>
      rate_limit(to = 'BobsPC', 
                content = 'video', 
                bandwidth = 65536)
\end{Verbatim}

The example above is a variation on the video bandwidth limitation example. In this case, we only care if Bob's PC is downloading more then 10 gigabytes of video traffic in a month. If it does reach this limit, then we want to limit the bandwidth to 64 kilobytes per second.

\begin{Verbatim}[fontsize=\small]
  match_AS(pkt.srcip, AS = '12345')) >>
      drop()
\end{Verbatim}

The second example is simpler. We are see if a packet came from a AS 12345. If so, we want to drop it. Implicitly, if we do not drop a packet, it will be forwarded.
\end{comment}
